
\chapter{Inherentno neuređeni proteini} % Main chapter title

\label{IDP} % For referencing 


Funkcionalni proteini sa delimičnim ili potpunim izostankom strukture (pri
fiziološkim uslovima) nalaze se svuda u živom svetu, do te mere da ima više
smisla upitati ''gde se oni ne nalaze?'' nego obratno \parencite{uversky2016}.
Danas je neuređenost proteina uzrokovala nastanak velikog broj hipoteza, od
$D^2$ koncepta bolesti \parencite{Uversky2008} pa sve do evolucije
višećelijskih organizama \parencite{Romero2006} i osobina prvih oblika života
\parencite{uversky2016, Trifonov2000}. Zajedno sa početkom 21. veka broj
naučnih radova koji se bave ovom temom doživljava skoro eksponencijalan
porast \parencite{oldfield2014}, ali  da bi se razumela popularnost i 
perspektiva koju polje donosi neophodno je osvrnuti se na istoriju.

Fišerova\footnote{ Emil Fišer bio je nemački hemičar koji je 1894. predložio
analogiju brave i ključa opisujući karakteristike enzima pivske
plesni \parencite{dunker2001}.  } analogija o \keyword{bravi i ključu} ponovo
otkrivena nezavisnim istraživanjima Hsien Wu,  Mirski i Paulinga 
postavila je temelje opšteprihvaćene
\keyword{struktura-funkcija} paradigme \parencite{dunker2001}.
Ova paradigma predlaže da su funkcionalne (operativne) osobine proteina
posledica njihovog jedinstvenog savijenog oblika (konformacije) tj.
% ''
% Specifične osobine nativnih\ref{} proteina mi pripisujemo njihovoj jedinstveno
% definisanoj konfiguraciji.  Denaturisane proteine mi smatramo okarakterisane
% izostankom jedinstveno definisane konfiguracije
% ''(Mirski i Pauling)
\textit{“the characteristic specific properties of native\footnote{ nativno stanje
  proteina je savijeno, operativno, funkcionalno stanje \parencite{dunker2001}.
Ovaj termin bio je isprepleten sa paradigmom struktura-funkcija} proteins we
attribute to their uniquely defined configurations”} \parencite{MirskyPauling1936}. Predloženi model prilagođen je
funkcionisanju enzima, čija sposobnost da katalizuju zavisi od jasno
definisanog geometrijskog oblika koji moraju da zauzmu odnosno u koji moraju da
se saviju.  Supstrat (ključ ili funkcija) diktira oblik enzima (brave ili
strukture) \parencite{biology}.  Kontrapozicijom sledi da nedostatak strukture
vodi izostanku funkcije.

Prvi kontraprimer navedene teorije javio se još 1950. Protein krvne plazme, serum
albumin pokazivao je veliku mogućnost vezivanja za različite
partnere \parencite{dunker2001}. Ovo otkriće ukazivalo je da specifične zahteve
enzima ne treba generalizovati na sve proteine. Ipak model brave i ključa i
njena poboljšana varijanta, \keyword{teorija indukovanog fita}\footnote{ Teorija
indukovanog fita omekšava rigidnost modela brava-ključ sugerišući da interakcija
sa supstratom indukuje konačni oblik enzima maksimizujući
reakciju \parencite{biology}} \en{induced-fit theory} dominirale su krajem
prošlog veka, zanemarujuću konstantno rastući skup funkcionalnih
''ne-nativnih'' proteina čije postojanje nisu mogle da objasne. Sa druge strane
tehnološki napreci u razlučivanju strukture proteina jasno su demonstrirali
obimno postojanje funkcionalnih proteina bez uređene 3D strukture (pri
fiziološkim uslovima)  od kojih su neki bili neuređeni celom
dužinom \parencite{dunker2001}.  Nova paradigma je bila neophodna.

Hipoteza proteinskog trojstva \parencite{dunker2001} (nastala tek početkom
21. veka) predlaže da funkcija proteina može zavisiti od bilo kog od tri
stanja ili tranzicije između tih stanja. Predložena stanja predstavljaju
nativne oblike proteina i analogna su najčešćim stanjima materije na zemlji.
Model je naknadno dopunjen još jednim stanjem:
\begin{enumerate}
  \item \keyword{Uređen protein} - čvrsto stanje

  \item \keyword{Topljiva globula} \en{molten globule} - tečno stanje

  \item \keyword{Pre-topljiva globula} \en{pre-molten globule} - međustanje\\ 
    Usled nejasne tranzicije između stanja topljivog globula i nasumičnog
    klupka (suprotno analogiji tečnog i gasovitog stanja\parencite{dunker2001})
    model je dopunjen.

  \item \keyword{Nasumično klupko} \en{random coil} - gasovito stanje
\end{enumerate}

Povezanost sekvence sa strukturom sugeriše da je neuređenost enkodirano
inherentno\footnote{ Inherentno ili prirođeno, nasleđeno} svojstvo \parencite{dunker2001}
stoga ove proteine nazivamo \keyword{inherentno\footnote{ U nedostatku
    adekvatne domaće reči koristimo najbliži sinonim reči \en{intrinsic} tj.
\en{inherent}, koja čuva suštinu originalnog značenja.  } neuređeni proteini}
\en{Intrinsically Disorderd Proteins} skraćeno \keyword{IDP}, a njihove neuređene
ali funkcionalne regione \keyword{IDPr} \parencite{uversky2016}. U ovom radu pod
neuređenošću proteina podrazumevaćemo inherentnu neuređenost osim ako to nije
drugačije naglašeno\footnote{ tumačenje neuređenosti zavisi od konteksta i može
  da označava denaturisane ili na drugi način dobijene nefunkcionalne
proteine}.

Današnje procene zastupljenosti pokazuju da 19\% aminokiselina kod
eukariota, 6\% kod bakterija i 4\% kod arhea pripadaju
IDPr \parencite{Peng2014}.  Čak 50\% proteina eukariota ima bar jedan IDPr duži
ili jednak 30 uzastopnih AK \parencite{Xue2012} dok je za 6\% do 17\%
predviđeno da su neuređeni celom dužinom \parencite{Tompa2002}.  Ovi podaci bude
veliko interesovanje naučnika da istraže funkciju i ponašanje IDP i IDPr.

\section {Osobine i uticaj na funkciju}

Detaljno opisivanje osobina i posledica neuređenosti proteina prevazilazi obim
rada zalazeći u biohemiju i biofiziku. Sa druge strane, količina novih saznanja
raste veoma brzo. Na primer, u časopisu \textit{Nature} objavljen je rad
\parencite{rebecca2018} koji kratko sumira najnovija saznanja koja
fundamentalno menjaju poglede na mogućnost jakog vezivanja potpuno neuređenih
proteina u dinamične komplekse.  Iz tih razloga navodimo samo globalne osobine
IDP i IDPr kao i osobine relevantne za ovo istraživanje.

\begin{itemize}

  \item
    Neuređenost je inherentno svojstvo sekvence \parencite{dunker2001}.
    Pokazano je da nisko očekivanje indeksa hidropatije\footnote{mera hidrofobnosti} zajedno sa visokim
    ukupnim naelektrisanjem predstavlja bitan preduslov koji sprečava savijanje
    proteina u fiziološkim uslovima \parencite{uversky2016}. Statističkom
    analizom otkriveno je klasterovanje aminokiselina u one koje promivišu
    uređenost C, W, I, Y, F, L, M, H i N \en{order promoting} i one koje
    promovišu neuređenost P, E, S, Q i K \en{disorder promoting}.
    \parencite{uversky2016, oldfield2014}. Opisane osobine daju validnost
    primeni mašinskog učenja u predviđanju neuređenih regiona proteina
    \parencite{oldfield2014}.

  \item
    Post translacione modifikacije proteina (PTM) značajno utiču na  kontrolu i
    proširenje funkcije pogotovo neuređenih delova proteina. Postoji značajno
    preklapanje gore pomenute klasifikacije aminokiselina sa skupom AK koje su
    često modifikovane \parencite{uversky2016}. Iako su PTM povezane sa
    neuređenošću i sugeriše veliki uticaj na funkciju proteina
    \parencite{uversky2016}, kompleksnost ove teme prevazilazi obime ovog
    istraživanja.

      % sadrže manje informacija, manji  Šenonov indeks.\\ Šenonov indeks $I = - \sum^n_{i=1} p_i \cdot log_2{p_i} $

  \item
    IDP i IDPr su po zastupljenosti AK prostije\footnote{ Prostije u smislu da
      sadrže manje informacija, manji  Šenonov indeks (mera kvaliteta
    informacije poistovećena sa brojem bita potrebnih da je kodiraju) }
    sekvence u poređenju sa domenima savijenih proteina. Ipak, usled manje
    restrikcija (obaveznog savijanja) mogućnost interakcije sa više partnera je
    mnogo veća što moguće funkcije čini raznolikim \parencite{uversky2016}.
    Pomenuta interakcija kod nekih neuređenih proteina vodi do njihovog
    potpunog ili parcijalnog savijanja dok neki i dalje ostaju neuređeni
    \parencite{uversky2016}.  Primena izreke \textit{manje je više} proizvela
    je brave koje otključava nekoliko ključeva i ključeve koji otključavaju
    nekoliko brava.

  \item 
    IDP i IDPr teško je strukturno kategorizovati \parencite{oldfield2014,
      dunker2001} iako su rani pokušaji napravljeni u radu \parencite{dunker2001}.
      Najopštiji opis strukture ovih proteina dat je kao
      \keyword{kombinacija različitih tipova foldona}\footnote{ Zbog nove
        prirode termina i manjka prevedene literature autor je odlučio da
      usvoji naziv u originalu.} \parencite{uversky2016}:
    \begin{itemize}
      \item \keyword{foldon} \en{foldon} je nezavisno organizujuća jedinica (region) proteina.
      \item \keyword{indukativni foldon} \en{inducible foldon} je IDPr koji savijanje lanca proteina postiže barem delom vezivajući se za partnera. 
      \item \keyword{ne-foldon} \en{non-foldon} je IDPr koji nikad ne postiže uređenost.
      \item \keyword{polu-foldon} \en{semi-foldon} je IDPr koji ostaje polovično neuređen i nakon vezivanja za partnera.
      \item \keyword{anti-foldon} \en{unfoldons} je region proteina koji iz uređenog prelazi u neuređeno stanje u cilju izvršavanja funkcije.
    \end{itemize}

  \item 
    Gore pomenut opšti prikaz strukture nastao je iz raznih opažanja
    interakcije, prvenstveno vezivanja proteina za partnere.
    % Detaljan opis osobina sa iscrpnom list opisana je u  radovima
    % \parencite{a2z, uversky2016}
    Detaljan opis i iscrpna lista ovih i drugih pojava može se naći u radovima
    \cite{uversky2016, a2z,  Tompa2009}.
    % kao i poglavljima 10, 12 i 14 knjige
    % \textit{Structure and Function of Intrinsically Disordered Proteins by
    % Peter Tompa, Alan Fersht}.

\end{itemize}

\section{Eksperimentalno ispitivanje neuređenosti}

Postoji veliki broj eksperimentalnih metoda za karakterizaciju strukture i
osobina proteina.  Svaka od njih ima prednosti, mane i nivo pouzdanosti. Da bi
se protein potpuno okarakterisao korisno je sagledati rezultate nekoliko
eksperimentalnih metoda. Isto važi i za karakterizaciju neuređenih regiona
proteina. \textit{\keyword{DisProt}} \cite{Piovesan2016} baza neuređenih
proteina verzija 7 na svojoj veb stranici nabraja čak 36 eksperimentalnih
tehnika.  Eksperimentalne tehnike koje se najčešće koriste za karakterizaciju
neuređenosti proteina su \cite{dunker2001}: 

\begin{itemize}
  \item Kristalografija X zracima \en{X-ray crystallography} 
  \item Spektrosokopija Nuklearnom Magnetnom Rezonancom (NMR) \en{NMR spectroscopy}
  \item Cirkularni dihroizam \en{Circular dichroism (CD) spectroscopy}
  \item Senzitivnost na proteolizu \en{Sensitivity to proteolysis}
  % \item \en{Protease digestion}
  % \item \en{Stoke’s radius determination}
\end{itemize}


\section{Predikcija neuređenosti}

Do danas napravljeno je preko 60 prediktora inherentno neuređenih proteina
\parencite{Meng2017}. Prediktor u kontekstu proteina je program koji 
računarskim metodama predviđa osobine proteina. Primer računarske metode
predstavljaju tehnike \keyword{mašinskog učenja} (skraćeno ML).  U radu
\parencite{Meng_c2017} hronološkim redosledom prikazane su karakteristike i
dostupnost tridesetak popularnih prediktora neuređenosti.


Istorijski posmatrano razlikujemo tri epohe razvoja \parencite{Meng_c2017}:
\begin{itemize}
  \item Prva generacija (1979\footnote{
      Nakon 1979. godine drugi prediktor nastao je tek 1997.
      \parencite{Meng_c2017} }-2001)\\
    Prvi prediktori oslanjali su se na razne fizičko-hemijske osobine proteina
    uključujući i osobine aminokiselina. 

  \item Druga generacija (2002-2006)\\
    Ovaj period okarakterisan je korišćenjem relativno jednostavnih
    prediktivnih modela: prediktore isključivo zasnovane na osobinama AK
    sekvence ali i popularne ML metode. Kao ulaz, pored sekvence, neke metode su
    podržavale i evolutivne profile.

  \item Treća generacija (2007-)\\
    Prediktori današnjice koriste komplikovanije ML modele. Uglavnom  se
    podrazumeva meta-prediktor koji kombinuju rezultate nekoliko običnih ML
    modela. Na primer, kombinacija NN, SVM i K-najbližih suseda tehnikom
    glasanja.

\end{itemize}


% Po arhitekturi prediktore delimo u četiri kategorije: \parencite{Meng_c2017}
% \begin{enumerate}
%   \item
%     scorring function based
%     % \en{
%     % These approaches input properties computed directly from the protein
%     % sequence, such as sequence alignment and propensity for intrachain
%     % interactions and binding, as well as the propensity for intrinsic disorder
%     % into a scoring function to predict disordered protein binding regions
%     % }
%
%   \item
%     ML metode 
%
%   \item
%     Meta-prediktori
%
%   \item
%     Predikcije na osnovu strukture\footnote{podrazumeva predviđanje strukturnih
%     elemenata proteina čije odsustvo predviđa neuređenost}.
% \end{enumerate}
%
%
% \subsection{Evaluacija ML modela}
% TODO, samo osnovne formule za preciznost i druge mere...


\subsection{PONDR familija prediktora i VSL2b}
\label{VSL2b}

PONDR familija \en{Predictors of Natural Disordered Regions} je grupa
prediktora druge generacije zasnovanih na neuronskim mrežama, kraće NN .
Neuronske mreže sa propagacijom unapred \en{feed forward NN} sa veličinom
prozora između 9 i 21 AK trenirane su na različitim trening skupovima
proteinskih sekvenci.  Finalni prediktor predstavlja kombinaciju nekoliko
neuronskih mreža od kojih je svaka specijalizovana za regione određene dužine
ili položaja.  PONDR familija sadrži nekoliko prediktora koji  se razlikuju 
po izboru trening skupova.
Oznaka ''VSL'' kodira tipove i poreklo atributa proteinskih trening skupova.
\begin{itemize}
  \item V - Opisuje eksperimentalnu tehniku kojom je neuređenost utvrđena na
    trening skupu \en{X-ray, NMR, circular dichroism}
  \item S - Prediktor je treniran na skupu proteina sa \keyword{kratkim}
      neuređenim regionima ($<30$ AK)
  \item L - Prediktor je treniran na skupu proteina sa \keyword{dugim}
    neuređenim regionima ($>30$ AK)
\end{itemize}

CASP \en{Critical Assessment of protein Structure Prediction} je takmičenje u
predikciji strukture proteina (ili neuređenosti) gde se objektivno ocenjuje
kvalitet razvijenih prediktora i počev od 1994. održava se svake dve godine.
Tokom CASP7 takmičenja 2006. VSL2b je evaluiran je kao prediktor sa ukupnim
najtačnijim predviđanjima neuređenosti \parencite{He2009}. Međutim, po
današnjim merilima \parencite{Meng2017} VSL2b ipak se smatra zastarelim.  Ali,
kako je VSL2b nezavistan paket koji se lako može pokrenuti na kućnom računaru i
projektovan je da bude brz (visoko propustan), ovo istraživanje temelji se
upravo na njemu.

VSL2b kao ulaz prima proteinsku sekvencu\footnote{
  Postoje varijante prediktora koje kao ulaz primaju evolutivni profil, ali zbog
  dodatne složenosti koraka PSI-BLAST pretrage ovaju pristup nije korišćen.
}
minimalne dužine 9 AK kodiranih jednim
karakterom i podržava azbuku od 20 standardnih AK. Rezultat predikcije je niz
ocena (verovatnoća) za svaku poziciju sekvence
% \footnote{ Autori često koriste termin ``ostatak`` \en{residue} kada misle na
% vrednost neke poziciju u sekvenci (polimeru).  Kod aminokiselina ``ostatak`` se
% odnosi na R grupu po kojoj razlikujemo aminokiseline.}
koje govore da li je pozicija uređena ili neuređena. Pozicija sa vrednostima iznad
0.5 smatra se neuređenim, a suprotno uređenim.






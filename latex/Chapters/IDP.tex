
\chapter{Inherentno neuređeni proteini} % Main chapter title

\label{IDP} % For referencing 


Funkcionalni proteini sa parcijalnim ili celovitim izostankom strukture (pri
fiziološkim uslovima) nalaze se svuda, do te mere da više ima smisla pitati
''gde se oni ne nalaze?'' nego obratno\parencite{uversky2016}.
Danas neuređenost proteina inspiriše veliki broj hipoteza, od $D^2$ koncepta
bolesti\parencite{d2uversky2008} sve do evolucije višećelijskih
organizama\parencite{romero2006} i osobina prvih oblika života
\parencite{trifonov2000, uversky2016}. Šta više, sa početkom 21. veka broj
naučnih radova koji se bave ovom temom prati skoro eksponencijalan
porast\parencite{oldfield2014} ali  da bi se razumela popularnost i sveža
perspektiva koju polje donosi neophodno je osvrnuti se na istoriju.

Fišerova\footnote{ Emil Fišer bio je Nemački hemičar koji je 1894. predložio analogiju brave i ključa opisujući
  karakteristike enzima pivske plesni\parencite{dunker2001}.
} analogija o \keyword{bravi i ključu} ponovo otkrivena nezavisnim
istraživanjima naučnika Hsien Wu,  Mirski i Paulinga (pedesetih godina prošlog
veka) postavila je temelje ''opšteprihvaćene'' \keyword{struktura-funkcija}
paradigme \parencite{dunker2001}.
% ''
% Specifične osobine nativnih\ref{} proteina mi pripisujemo njihovoj jedinstveno
% definisanoj konfiguraciji.  Denaturisane proteine mi smatramo okarakterisane
% izostankom jedinstveno definisane konfiguracije
% ''(Mirski i Pauling)
\en{“The characteristic specific properties of native\footnote{ nativno stanje
proteina je savijeno, operativno, funkcionalno stanje\parencite{dunker2001}.
Ovaj termin bio je isprepleten sa struktura-funkcija paradigmom }
proteins we attribute to their uniquely defined configurations. The denatured
protein molecule we consider to be characterized by the absence of a uniquely
defined configuration.”} (Mirski i Pauling)
Predloženi model prilagođen je ponašanju enzima, čija sposobnost da
katalizuju\footnote{katalizacij podrazumeva ubrzavanje ili omogućavanje
hemijske reakcije nad substratom\parencite{biology}}  zavisi od jasno definisanog geometriskog
oblika koji moraju da zauzmu odnosno u koji moraju da se saviju.
Substrat\footnote{substrat je molekul nad kojim enzim deluje\parencite{biology}} (ključ ili
funkcija) diktira oblik enzima (brave ili strukture)\parencite{biology}.
Kontrapozicijom sledi da nedostatak strukture vodi izostanku funkcije.

Prvi kontraprimer \en{serum albumin} (1950) ukazivao je da specifične zahteve
enzima ne treba generalizovati na sve proteine. Ipak brava-ključ model i njena
poboljšana varijanta \keyword{teorija indukovanog fita}\footnote{
Teorija indukovanog fita omekšava rigidnost brava-ključ model sugerišući da 
interakcija sa substratom indukuje konačni oblik enzima maksimizujući reakciju\parencite{biology}}
\en{induced-fit theory}
dominirale su prošlim vekom, zanemarujuću konstantno rastući skup funkcionalnih
''ne-nativnih'' proteina čije postojanje nisu mogle da objasne. Sa druge strane
tehnološki napretci u razlučivanju strukture proteina jasno su demonstrirali
obimno postojanje funkcionalnih proteina bez uređene 3D strukture (pri
fiziološkim uslovima)  od kojih su neki bili neuređeni celom
dužinom\parencite{dunker2001}.
Nova paradigma je bila neophodan.

Hipoteza proteinskog ''trojstva''\parencite{dunker2001} (nastala tek početkom
stoleća) predlaže da funkcija proteina može zavisiti od bilo koje od ''tri''
stanja ili tranzicije između tih stanja. Predložena stanja predstavljaju
nativne oblike proteina i analogna su najčešćim stanjima materije na zemlji.
Model je naknadno dopunjen još jednim stanjem:
\begin{enumerate}
  \item \keyword{Uređen protein}, čvrsto stanje

  \item \keyword{Topljiva globula} \en{molten globule}, tečno stanje

  \item \keyword{Pre-topljiva globula} \en{pre-molten globule} međustanje.
    Usled nejasne tranzicije između stanja topljivog globula i nasumičnog klupka
    (suprotno analogiji tečnog i gasovitog stanja)\parencite{dunker2001} model je
    dopunjen.

  \item \keyword{Nasumično klupko} \en{random coil}, gasovito stanje
\end{enumerate}

Povezanost sekvence sa strukturom sugeriše da je neuređenost enkodirano
inherentno svojstvo \parencite{dunker2001} stoga ove proteine nazivamo
\keyword{Inherentno\footnote{
    U nedostatku adekvatne domaće reči koristimo najbliži sinonim reči
    \en{intrinsic} tj. \en{inherent}, koja čuva suštinu originalnog značenja.
} Neuređeni Proteini} \en{Intrinsically Disorderd Proteins}
skraćeno \keyword{IDP} a njiove neuređen ali funkcionalne regione
\keyword{IDPr}\parencite{uversky2016}. U ovom radu pod neuređenošću proteina
implicitno se podrazumeva inherentna neuređenost osim ako to nije drugačije
naglašeno\footnote{
tumačenje neuređenosti zavisi od konteksta i može da označava denaturisane ili na
drugi način dobijene nefunkcionalne proteine}.

Današnje procene zastupljenosti pronašle su da 19\% svih aminokiselina
kod eukariota, 6\% kod bakterija i 4\% kod areha pripadaju IDPr\parencite{peng2015b}.
Čak 50\% proteina eukariota ima bar jedan IDPr duži ili jednak od 30 uzastopnih AK\parencite{Xue2012}
dok je za između 6\% i 17\% predviđeno da su neurđenim celom dužinom\parencite{tompa2002}.
Ovi podaci bude veliko interesovanje naučnika da istraže funkciju i ponašanje
IDP i IDPr.

\section {Osobine i uticaj na funkciju}

Detaljno opisivanje osobina i posledica neuređenosti prevazilazi obim rada i
stručnost autora zalazeći u biohemiju i biofiziku. Takođe, maloistraženi
potencijala ove oblasti proizvodi veliku količinu novih saznanja. Samo za vreme
pisanja ovoga poglavlja, u časopisu \en{Nature}  objavljen je kratak
rad\parencite{rebecca2018} koji za neupućenog čitaoca fundamentalno menja
poglede na mogućnost jakog vezivanja visoko neređenih proteina u dinamične
komplekse.  Iz tih razloga navodimo samo globalne osobina IDP i IDPr kao i
osobina relevantne za naš rad.

\begin{itemize}

  \item
    Neuređenost je inherentno svojstvo sekvence\parencite{dunker2001}.
    Pokazano je da nisko očekivanje indeksa hidropatije zajedno sa visokim
    ukupnim nabojem predstavlja bitan preduslov koji sprečava savijanje proteina
    u fiziološkim uslovima \parencite{uverksy}. Statističkom analizom otkriveno
    je klasterovanje aminokiselina u one koje promivišu uređenje C, W, I, Y, F,
    L, M, H i N \en{order promoting} i one koje promovišu neuređenost P, E, S,
    Q i K \en{disorder promoting}. \parencite{oldfield2014, uversky2016}
    Opisane osobine daju validnost primeni mašinskog učenja u predviđanju
    neurđenih regiona proteina \parencite{oldfield2014}.

  \item
    PTM proteina značajno utiču na  kontrolu i proširenje funkcije pogotovo
    neuređenih delova proteina. Postoji značajno preklapanje gorepomenute
    klasifikacije aminokiselina sa skupom AK koje su često modifikovane
    \parencite{uversky2016}. Iako je PTM povezano sa neuređenošću i sugeriše
    beskrajne uticaje na funkciju proteina\parencite{uversky2016} kompleksnost
    ove teme prevazilazi obime ovog istraživanja.

  \item
    IDP i IDPr su po zastupljenosti AK prostije\footnote{ prostije u smislu da
    sadrže manje informacija(Šenonov indeks)} sekvence u poređenju sa domenima
    savijenih proteina. Ipak usled manje restrikcija (obaveznog savijanja)
    mogućnost interakcije sa više partnera je mnogo veća što moguće funkcije
    čini raznolikim \parencite{uversky2016}.  Pomenuta interakcija kod nekih
    neuređenih proteina vodi do njihovog potpunog, parcijalnog savijanja dok
    neki idalje ostaju neuređeni \parencite{uversky2016}.  Bukvalna evoluciona
    primena izreke ''manje je više'' proizvela je brave koje otključava
    nekoliko ključeva i ključeve koji otključavaju nekoliko brava.

  \item 
    IDP i IDPr teško je strukturno kategorizovati \parencite{dunker2001,
    oldfield20014}(neki pokušaji su napravljeni \parencite{dunker2001}). 
    Najuopšteniji opis strukture ovih proteina dat je kao \keyword{kombinacija
      različitih tipova foldona}\footnote{ Zbog nove prirode termina i
      manjka prevedene literature autor je odlučio da usvoji naziv u originalu.}\parencite{uversky2016}:
    \begin{itemize}
      \item \keyword{foldon} \en{foldon} je nezavisno organizujuća jedinica (region) proteina.
      \item \keyword{indukativni foldon} \en{inducible foldon} je IDPr koji savijanje postiže barem delom vezivajući se za partnera. 
      \item \keyword{ne-foldon} \en{non-foldon} je IDPr koji nikad ne postiže uređenje.
      \item \keyword{polu-foldon} \en{semi-foldon} je IDPr koji ostaje polovično neuređen i nakon vezivanja za partnera.
      \item \keyword{anti-foldon} \en{unfoldons} je region proteina koji iz uređenog prelazi u neuređeno stanje u cilju vršenja funkcije.
    \end{itemize}

  \item 
    Gore pomenut opšti prikaz strukture nastao je iz raznih opažanja
    interakcije, prvestveno vezivanja proteina za partnere. Za detaljan opis i
    iscrpnu listu ovih i drugih pojava preporučujemo čitanje \parencite{a2z,
    uversky2016} kao i poglavlja 10, 12 i 14 iz knjige \en{Structure and Function of
  Intrinsically Disordered Proteins by Peter Tompa, Alan Fersht}.

\end{itemize}

\section{Funkcija}
Funkcija proteina može biti sagledana iz barem tri ugla: molekulske funkcije,
biološkog procesa kome pripada i lokacije u ćeliji gde se funkcija odvija
\parencite{go2000}(postoje i drugi sistemi klasifikacije\ref{keyword}).
Kako je cilj ovog rada molekulska funkcija tabelom\ref{tab:funkcija_uvod} ukratko
navodimo (bez poredka) ustanovljen\parencite{Xie2007} molekulske funkcije koje
se pripisuju (ne)uređenju proteina. Ovo su takođe rezultati za koje se nadamo
da će naše istraživanje potvrditi.

\begin{table}[h!]
  \centering
  \caption{Odnos molekulske funkcije i uređenosti}
  \label{tab:funkcija_uvod}
  \begin{tabular}{c}
  
  \end{tabular}
\end{table}


\keyword{Ovo možda za kraj ostaviti...} \\
Novija istraživanja nad eksperimentalno dokazanim IDP i IDPr dovela su do kreiranja
ontologija(po ugledu na GO\parencite{GO2000}) za opis funkcija neuređenih
proteina. Ontologije su sastavni deo DisProt\parencite{disprot7} baze
eksperimentalno dokazanih IDP i IDPr... novi prediktori postoje...

\section{Eksperimentalno ispitivanje neuređenosti}

\begin{itemize}
  \item \keyword{Kristalografija X zracima} \en{X-ray crystallography}
  \item Spektrosokopija Nuklearnom Magnetnom Rezonancom (NMR) \en{NMR spectroscopy}
  \item \en{Circular dichroism (CD) spectroscopy}
  \item \en{Protease digestion}
  \item \en{Stoke’s radius determination}
\end{itemize}


\section{Predikcija neuređenosti}

Do danas napravljeno je preko 60 prediktora inherentno neuređenih
proteina\parencite{meng2017}. U radu\parencite{meng_c2017} hronološkim
redosledom prikazane su karakteristike i dostupnost tridesetak popularnih
prediktora.


Istoriski posmatrano razlikujemo tri epohe razvoja:\parencite{meng_c2017}
\begin{itemize}
  \item Prva generacija (1979\footnote{
      Nakon 1979 drugi (prvi ozbiljni) prediktor nastao je tek 1997.\parencite{meng_c2017}
    }-2001)
    Prvi prediktori oslanjali su se na razne fizčko hemijske osobine proteina
    uključujući i osobine\ref{} aminokiselina: 

  \item Druga generacija (2002-2006)\\
    Ovaj period okarakterisan je korišćenjem relativno jednostavnih
    prediktivnih ML modela koji koriste isključivo svojstava AK ulazne
    sekvence\footnote{ Takođe javljaju se prediktori koji koriste evolutivne
      profile sekvence (PSSM skoring matrice) dobijene PSI-BLAST pretragom}.

  \item Treća generacija (2007-)\\
    Prediktori današnjice koriste komplikovanije ML modele. Uglavnom  se
    podrazumeva meta-prediktor koji kombinuju rezultate nekoliko običnih ML
    modela. Recimo kombinacija NN, SVM i K-najblilžih suseda tehnikom
    glasanja.

\end{itemize}


Po arhitekturi predikotre delimo u četiri kategorije:\parencite{meng_c2017}
\begin{enumerate}
  \item
    scorring function based
    % \en{
    % These approaches input properties computed directly from the protein
    % sequence, such as sequence alignment and propensity for intrachain
    % interactions and binding, as well as the propensity for intrinsic disorder
    % into a scoring function to predict disordered protein binding regions
    % }

  \item
    ML metode 

  \item
    Meta-prediktori

  \item
    Predikcije na osnovu strukture\footnote{podrazumeva predviđanje strukturnih
    elemenata proteina čije odsustvo predviđa neuređenost}.
\end{enumerate}


\subsection{Evaluacija ML modela}
TODO, samo osnovne formule za prciznost i druge mere...


\subsection{PONDR familija prediktora i VSL2b}
\label{VSL2b}

PONDR familija \en{Predictors of Natural Disordered Regions} je grupa
prediktora druge generacije zasnovanih na neuronskim mrežama, kraće NN .
Neuronske Mreže sa propagacijom unapred \en{feed forward NN} sa veličinom
prozora između 9 i 21 AK trenirane su nad različitim trening skupovima
proteinskih sekvenci.  Finalni prediktor predstavlja kombinaciju nekoliko
neuronskih mreža od kojih je svaka specijalizovana za regione određene dužine
ili položaja.  PONDR familija ima nekoliko prediktora koji  se razlikuju u
načinu treniranja što je postignuto kombinacijom pomenutih trening skupova.
Oznaka ''VSL2b'' kodira tipove i poreklo atributa proteinskih trening skupova.
\begin{itemize}
  \item V - Opisuje eksperimentalnu tehniku kojom je neurđenost utvrđena na
    trening skupu \en{X-ray, NMR, circular dichroism}
  \item S - Prediktor je treniran na skupu proteina sa \keyword{kratkim}
      neruređenim regionim ($<30$ AK)
  \item L - Prediktor je treniran na skupu proteina sa \keyword{dugim}
    neuređenim regionima ($>30$ AK)
\end{itemize}

Tokom CASP7 takimčenja 2008. VSL2b je evaluiran kao prediktor sa ukupnim najtačnijim predviđanjima\parencite{bohe2009}.
.Međutim po današnjim merilima \parencite{meng2017} VSL2b ipak se smatra
zastarelim.  Ali, kako je VSL2b nezavistan paket koji se lako može pokrenuti na
kućnom računaru i projektovan je da bude brz (visoko propustan) naše
istraživanje temelji se na njemu.

VSL2b kao ulaz prima proteinsku sekvencu\footnote{
Ulaz VSL2b može biti i evolutivni profil što poboljšava rezultat,  međutim zbog
dodatnog koraka PSI-BLAST pretrage ovaju pristup nije korišćen.
(posledice ???)
}
minimalne dužine 9 AK kodiranih jednim
karakterom. Podržava azbuku od samo 20 standardnih AK.  Izlaz je niz
ocena (verovatnoća) za svaku aminokiselinu da li pripada neuređenom regionu to
jest da je taj rezidual
\footnote{ Rezidual je čest naziv koji se koristi za aminokiseline i nukleinske
  kiseline na nekoj poziciji sekvence.  Naziv potiče od hemijskih tehnika
  prečišćavanja čiji su rezultati reziduali (ostaci).
} neuređen. Reziduale sa vrednostima iznad 0.5 smatra se neuređenim a suprotno
uređenim.






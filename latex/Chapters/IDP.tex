
\chapter{Inherentno neuređeni proteini} % Main chapter title

\label{IDP} % For referencing 


Funkcionalni proteini sa delimičnim ili potpunim izostankom strukture (pri
fiziološkim uslovima) nalaze se svuda u živom svetu, do te mere da ima više
smisla upitati ''gde se oni ne nalaze?'' nego obratno \parencite{uversky2016}.
Danas neuređenost proteina je uzrokovala nastanak velikog broj hipoteza, od
$D^2$ koncepta bolesti \parencite{d2uversky2008} pa sve do evolucije
višećelijskih organizama \parencite{romero2006} i osobina prvih oblika života
\parencite{trifonov2000, uversky2016}. Šta više, sa početkom 21. veka broj
naučnih radova koji se bave ovom temom doživljava skoro eksponencijalan
porast \parencite{oldfield2014},ali  da bi se razumela popularnost i 
perspektiva koju polje donosi neophodno je osvrnuti se na istoriju.

Fišerova\footnote{ Emil Fišer bio je Nemački hemičar koji je 1894. predložio
analogiju brave i ključa opisujući karakteristike enzima pivske
plesni \parencite{dunker2001}.  } analogija o \keyword{bravi i ključu} ponovo
otkrivena nezavisnim istraživanjima Hsien Wu,  Mirski i Paulinga (pedesetih
godina prošlog veka) postavila je temelje ''opšteprihvaćene''
\keyword{struktura-funkcija} paradigme \parencite{dunker2001}.
% ''
% Specifične osobine nativnih\ref{} proteina mi pripisujemo njihovoj jedinstveno
% definisanoj konfiguraciji.  Denaturisane proteine mi smatramo okarakterisane
% izostankom jedinstveno definisane konfiguracije
% ''(Mirski i Pauling)
\en{“The characteristic specific properties of native\footnote{ nativno stanje
  proteina je savijeno, operativno, funkcionalno stanje \parencite{dunker2001}.
Ovaj termin bio je isprepleten sa paradigmom struktura-funkcija} proteins we
attribute to their uniquely defined configurations. The denatured protein
molecule we consider to be characterized by the absence of a uniquely defined
configuration.”} (Mirski i Pauling) Predloženi model prilagođen je
funkcionisanju enzima, čija sposobnost da katalizuju\footnote{katalizacija
podrazumeva ubrzavanje ili omogućavanje hemijske reakcije sa
substratom \parencite{biology}}  zavisi od jasno definisanog geometriskog oblika
koji moraju da zauzmu odnosno u koji moraju da se saviju.
Substrat\footnote{substrat je molekul sa kojim enzim
deluje \parencite{biology}} (ključ ili funkcija) diktira oblik enzima (brave ili
strukture) \parencite{biology}.  Kontrapozicijom sledi da nedostatak strukture
vodi izostanku funkcije.

Prvi kontraprimer navedene teorije javio se još 1950. Protein krvne plazme, serum
albumin pokazivao je veliku mogućnost vezivanja za različite
partnere \parencite{dunker2001}. Ovo otkriće ukazivalo je da specifične zahteve
enzima ne treba generalizovati na sve proteine. Ipak model brave i ključ i
njena poboljšana varijanta \keyword{teorija indukovanog fita}\footnote{ Teorija
indukovanog fita omekšava rigidnost modela brava-ključ sugerišući da interakcija
sa substratom indukuje konačni oblik enzima maksimizujući
reakciju \parencite{biology}} \en{induced-fit theory} dominirale su krajem
prošlog veka, zanemarujuću konstantno rastući skup funkcionalnih
''ne-nativnih'' proteina čije postojanje nisu mogle da objasne. Sa druge strane
tehnološki napretci u razlučivanju strukture proteina jasno su demonstrirali
obimno postojanje funkcionalnih proteina bez uređene 3D strukture (pri
fiziološkim uslovima)  od kojih su neki bili neuređeni celom
dužinom \parencite{dunker2001}.  Nova paradigma je bila neophodan.

Hipoteza proteinskog ''trojstva'' \parencite{dunker2001} (nastala tek početkom
21. veka) predlaže da funkcija proteina može zavisiti od bilo kojeg od ''tri''
stanja ili tranzicije između tih stanja. Predložena stanja predstavljaju
nativne oblike proteina i analogna su najčešćim stanjima materije na zemlji.
Model je naknadno dopunjen još jednim stanjem:
\begin{enumerate}
  \item \keyword{Uređen protein} - čvrsto stanje

  \item \keyword{Topljiva globula} \en{molten globule} - tečno stanje

  \item \keyword{Pre-topljiva globula} \en{pre-molten globule} - međustanje\\ 
    Usled nejasne tranzicije između stanja topljivog globula i nasumičnog
    klupka (suprotno analogiji tečnog i gasovitog stanja) \parencite{dunker2001}
    model je dopunjen.

  \item \keyword{Nasumično klupko} \en{random coil} - gasovito stanje
\end{enumerate}

Povezanost sekvence sa strukturom sugeriše da je neuređenost enkodirano
inherentno\footnote{ Inherentno ili prirođeno, nasleđeno} svojstvo \parencite{dunker2001}
stoga ove proteine nazivamo \keyword{Inherentno\footnote{ U nedostatku
    adekvatne domaće reči koristimo najbliži sinonim reči \en{intrinsic} tj.
\en{inherent}, koja čuva suštinu originalnog značenja.  } Neuređeni Proteini}
\en{Intrinsically Disorderd Proteins} skraćeno \keyword{IDP}, a njihove neuređene
ali funkcionalne regione \keyword{IDPr} \parencite{uversky2016}. U ovom radu pod
neuređenošću proteina podrazumevaćemo inherentnu neuređenost osim ako to nije
drugačije naglašeno\footnote{ tumačenje neuređenosti zavisi od konteksta i može
  da označava denaturisane ili na drugi način dobijene nefunkcionalne
proteine}.

Današnje procene zastupljenosti pronašle su da 19\% aminokiselina kod
eukariota, 6\% kod bakterija i 4\% kod arhea pripadaju
IDPr \parencite{peng2015b}.  Čak 50\% proteina eukariota ima bar jedan IDPr duži
ili jednak od 30 uzastopnih AK \parencite{Xue2012} dok je za 6\% do 17\%
predviđeno da su neurđeni celom dužinom \parencite{tompa2002}.  Ovi podaci bude
veliko interesovanje naučnika da istraže funkciju i ponašanje IDP i IDPr.

\section {Osobine i uticaj na funkciju}

Detaljno opisivanje osobina i posledica neuređenosti prevazilazi obim rada
zalazeći u biohemiju i biofiziku. Sa druge strane broj novih saznanja raste
jako brzo. Recimo, u časopisu \textit{Nature} objavljen je rad \parencite{rebecca2018} koji
kratko sumira najnovija saznanja koja fundamentalno menjaju poglede na
mogućnost jakog vezivanja potpuno neuređenih proteina u dinamične komplekse.
Iz tih razloga navodimo samo globalne osobine IDP i IDPr kao i osobine
relevantne za naš rad.

\begin{itemize}

  \item
    Neuređenost je inherentno svojstvo sekvence \parencite{dunker2001}.
    Pokazano je da nisko očekivanje indeksa hidropatije\footnote{mera hidrofobnosti} zajedno sa visokim
    ukupnim nabojem predstavlja bitan preduslov koji sprečava savijanje
    proteina u fiziološkim uslovima \parencite{uversky2016}. Statističkom
    analizom otkriveno je klasterovanje aminokiselina u one koje promivišu
    uređenost C, W, I, Y, F, L, M, H i N \en{order promoting} i one koje
    promovišu neuređenost P, E, S, Q i K \en{disorder promoting}.
    \parencite{oldfield2014, uversky2016} Opisane osobine daju validnost
    primeni mašinskog učenja u predviđanju neuređenih regiona proteina
    \parencite{oldfield2014}.

  \item
    Post translacione modifikacije proteina (PTM) značajno utiču na  kontrolu i
    proširenje funkcije pogotovo neuređenih delova proteina. Postoji značajno
    preklapanje gore pomenute klasifikacije aminokiselina sa skupom AK koje su
    često modifikovane \parencite{uversky2016}. Iako je PTM povezano sa
    neuređenošću i sugeriše velike uticaje na funkciju proteina
    \parencite{uversky2016} kompleksnost ove teme prevazilazi obime ovog
    istraživanja.

  \item
    IDP i IDPr su po zastupljenosti AK prostije\footnote{ prostije u smislu da
    sadrže manje informacija (Šenonov indeks)} sekvence u poređenju sa domenima
    savijenih proteina. Ipak usled manje restrikcija (obaveznog savijanja)
    mogućnost interakcije sa više partnera je mnogo veća što moguće funkcije
    čini raznolikim \parencite{uversky2016}.  Pomenuta interakcija kod nekih
    neuređenih proteina vodi do njihovog potpunog ili parcijalnog savijanja dok
    neki i dalje ostaju neuređeni \parencite{uversky2016}.  Bukvalna evoluciona
    primena izreke ''manje je više'' proizvela je brave koje otključava
    nekoliko ključeva i ključeve koji otključavaju nekoliko brava.

  \item 
    IDP i IDPr teško je strukturno kategorizovati \parencite{dunker2001,
      oldfield20014} ali su (neki pokušaji su napravljeni \parencite{dunker2001}).
      Najuopšteniji opis strukture ovih proteina dat je kao
      \keyword{kombinacija različitih tipova foldona}\footnote{ Zbog nove
        prirode termina i manjka prevedene literature autor je odlučio da
      usvoji naziv u originalu.} \parencite{uversky2016}:
    \begin{itemize}
      \item \keyword{foldon} \en{foldon} je nezavisno organizujuća jedinica (region) proteina.
      \item \keyword{indukativni foldon} \en{inducible foldon} je IDPr koji savijanje postiže barem delom vezivajući se za partnera. 
      \item \keyword{ne-foldon} \en{non-foldon} je IDPr koji nikad ne postiže uređenost.
      \item \keyword{polu-foldon} \en{semi-foldon} je IDPr koji ostaje polovično neuređen i nakon vezivanja za partnera.
      \item \keyword{anti-foldon} \en{unfoldons} je region proteina koji iz uređenog prelazi u neuređeno stanje u cilju vršenja funkcije.
    \end{itemize}

  \item 
    Gore pomenut opšti prikaz strukture nastao je iz raznih opažanja
    interakcije, prvestveno vezivanja proteina za partnere.
    % Detaljan opis osobina sa iscrpnom list opisana je u  radovima \parencite{a2z, uversky2016}
    Detaljan opis i iscrpna listu ovih i drugih pojava može se naći u 
    \parencite{a2z, uversky2016} kao i poglavljima 10, 12 i 14 iz knjige \en{Structure and Function of
  Intrinsically Disordered Proteins by Peter Tompa, Alan Fersht}.

\end{itemize}

\section{Funkcija proteina}
Funkcija proteina može biti sagledana iz tri ugla: molekulske funkcije,
biološkog procesa kome pripada i lokacije u ćeliji gde se funkcija odvija
\parencite{go2000}(postoje i drugi sistemi klasifikacije\ref{keyword}).  Kako
je cilj ovog rada molekulska funkcija Tabelom\ref{tab:funkcija_uvod} ukratko
navodimo (bez poretka) ustanovljene \parencite{Xie2007} molekulske funkcije koje
se pripisuju (ne)uređenosti proteina. Ovo su takođe rezultati za koje se nadamo
da će naše istraživanje potvrditi.

\begin{table}[h!]
  \centering
  \caption{Odnos molekulske funkcije i uređenosti}
  \label{tab:funkcija_uvod}
  \begin{tabular}{c}
  
  \end{tabular}
\end{table}


\keyword{Ovo možda za kraj ostaviti...} \\
Novija istraživanja nad eksperimentalno dokazanih IDP i IDPr dovela su do
kreiranja ontologija(po ugledu na GO \parencite{GO2000}) za opis funkcija
neuređenih proteina. Ontologije su sastavni deo DisProt \parencite{disprot7}
baze eksperimentalno dokazanih IDP i IDPr... novi prediktori postoje...

\section{Eksperimentalno ispitivanje neuređenosti}

\begin{itemize}
  \item \keyword{Kristalografija X zracima} \en{X-ray crystallography}
  \item Spektrosokopija Nuklearnom Magnetnom Rezonancom (NMR) \en{NMR spectroscopy}
  \item \en{Circular dichroism (CD) spectroscopy}
  \item \en{Protease digestion}
  \item \en{Stoke’s radius determination}
\end{itemize}


\section{Predikcija neuređenosti}

Do danas napravljeno je preko 60 prediktora inherentno neuređenih
proteina \parencite{meng2017}. U radu \parencite{meng_c2017} hronološkim
redosledom prikazane su karakteristike i dostupnost tridesetak popularnih
prediktora.


Istorijski posmatrano razlikujemo tri epohe razvoja: \parencite{meng_c2017}
\begin{itemize}
  \item Prva generacija (1979\footnote{
      Nakon 1979 drugi prediktor nastao je tek 1997. \parencite{meng_c2017}
    }-2001)
    Prvi prediktori oslanjali su se na razne fizičko-hemijske osobine proteina
    uključujući i svojstva\ref{} aminokiselina: 

  \item Druga generacija (2002-2006)\\
    Ovaj period okarakterisan je korišćenjem relativno jednostavnih
    prediktivnih ML modela koji koriste isključivo svojstava AK ulazne
    sekvence.

  \item Treća generacija (2007-)\\
    Prediktori današnjice koriste komplikovanije ML modele. Uglavnom  se
    podrazumeva meta-prediktor koji kombinuju rezultate nekoliko običnih ML
    modela. Na primer kombinacija NN, SVM i K-najbližih suseda tehnikom
    glasanja.

\end{itemize}


Po arhitekturi predikotre delimo u četiri kategorije: \parencite{meng_c2017}
\begin{enumerate}
  \item
    scorring function based
    % \en{
    % These approaches input properties computed directly from the protein
    % sequence, such as sequence alignment and propensity for intrachain
    % interactions and binding, as well as the propensity for intrinsic disorder
    % into a scoring function to predict disordered protein binding regions
    % }

  \item
    ML metode 

  \item
    Meta-prediktori

  \item
    Predikcije na osnovu strukture\footnote{podrazumeva predviđanje strukturnih
    elemenata proteina čije odsustvo predviđa neuređenost}.
\end{enumerate}


\subsection{Evaluacija ML modela}
TODO, samo osnovne formule za preciznost i druge mere...


\subsection{PONDR familija prediktora i VSL2b}
\label{VSL2b}

PONDR familija \en{Predictors of Natural Disordered Regions} je grupa
prediktora druge generacije zasnovanih na neuronskim mrežama, kraće NN .
Neuronske mreže sa propagacijom unapred \en{feed forward NN} sa veličinom
prozora između 9 i 21 AK trenirane su na različitim trening skupovima
proteinskih sekvenci.  Finalni prediktor predstavlja kombinaciju nekoliko
neuronskih mreža od kojih je svaka specijalizovana za regione određene dužine
ili položaja.  PONDR familija sadrži nekoliko prediktora koji  se razlikuju 
po izboru trening skupova.
Oznaka ''VSL2b'' kodira tipove i poreklo atributa proteinskih trening skupova.
\begin{itemize}
  \item V - Opisuje eksperimentalnu tehniku kojom je neuređenost utvrđena na
    trening skupu \en{X-ray, NMR, circular dichroism}
  \item S - Prediktor je treniran na skupu proteina sa \keyword{kratkim}
      neruređenim regionim ($<30$ AK)
  \item L - Prediktor je treniran na skupu proteina sa \keyword{dugim}
    neuređenim regionima ($>30$ AK)
\end{itemize}

Tokom CASP7 takmičenja 2008. VSL2b je evaluiran kao prediktor sa ukupnim
najtačnijim predviđanjima \parencite{bohe2009}. Međutim, po današnjim merilima
\parencite{meng2017} VSL2b ipak se smatra zastarelim.  Ali, kako je VSL2b
nezavistan paket koji se lako može pokrenuti na kućnom računaru i projektovan
je da bude brz (visoko propustan) ovo istraživanje temelji se upravo na njemu.

VSL2b kao ulaz prima proteinsku sekvencu\footnote{
Ulaz VSL2b može biti i evolutivni profil što poboljšava rezultat,  međutim zbog
dodatnog koraka PSI-BLAST pretrage ovaju pristup nije korišćen.
(posledice ???)
}
minimalne dužine 9 AK kodiranih jednim
karakterom. Podržava azbuku od 20 standardnih AK.  Izlaz je niz
ocena (verovatnoća) za svaku poziciju sekvence
% \footnote{ Autori često koriste termin ``ostatak`` \en{residue} kada misle na
% vrednost neke poziciju u sekvenci (polimeru).  Kod aminokiselina ``ostatak`` se
% odnosi na R grupu po kojoj razlikujemo aminokiseline.}
koje govore da li je pozicija uređena ili neuređena. Pozicija sa vrednostima iznad
0.5 smatra se neuređenim, a suprotno uređenim.






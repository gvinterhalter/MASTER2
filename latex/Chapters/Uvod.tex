
\chapter{Uvod} % Main chapter title

\label{Uvod} % For referencing 


%------------------------------------------------------------------------------

\section{Osnovni biološki i hemiski koncepti}

Svi živi organizmi sastoje se od jedne ili više ćelija a svaka ćelija od
molekula. Veliki \footnote{ Obično se molekulska masa od $1000 Da$ (Daltona) uzima kao 
granica između malih molekula i makromolekula.}
molekuli (makromolekuli) biološkog porekla obično \footnote{
  Lipidi recimo nisu polimeri ali su principijalno slični
} su sačinjeni od
ponavljajućih strukturnih jedinica \keyword{monomera} \textit{(mono- = jedan,
mer- = deo)}, međusobno povezanih \keyword{kovalentnim} vezama.  Takav molekul
zovemo \keyword{polimer} \textit{(poli- =mnogo, -mer= deo)}. 
% Polimer može da bude \textit{homo-polimer}, sačinjen od jednog tipa monomera
% ili suprotno \textit{hetero-polimer}, sačinjen od nekoliko raznih tipova
% monomera.
Skup monomera možemo da smatramo azbukom koja gradi jezik polimera.  Mali broj
monomera je dovoljan za strukturnu kompleksnost bilo koje ćelije.  Tri 
najznačajnija tipa bioloških polimera i njihovi monomeri prikazani su tableom
\ref{tab:polimeri}.


\begin{table}[htpb]
  \centering
  \caption{Najznačajniji biološki polimeri}
  \label{tab:polimeri}
  \begin{tabular}{ll}
    \keyword{Polimer}            & \keyword{Monomer} \\
    Ugljeni hidrati              & Monosaharid (šećeri) \\
    Nukleinska kiselina (DNK)    & Nukleotid \\
    Protein                      & Aminokiselina \\
    \hline
  \end{tabular}
\end{table}



\label{sec:}
\subsection{Proteini}

Proteini su najčešći biološki makromolekuli koji čine i do $80\%$ suve mase
organizma.  Strukturno protein je linearan polimeri sačinjen od lanca
\keyword{aminokiselina}(monomeri). 


\section{Inherentno neuređeni proteini}

Funkcionalni proteini sa parcijalnim ili celovitim izostankom strukture nalaze
se svuda, do te mere da više ima smisla pitati ''gde se neuređeni proteini ne
nalaze?'' nego obratno\parencite{uversky2016}. Danas neuređenost proteina utiče
na veliki broj različitih hipoteza, od D2\parencite{} koncepta bolesti sve do
evolucije višećeliskih organizama \parencite{uversky2016}. Takođe od početka
stoleća broj naučnih radova koji se bave ovom temom prati skoro eksponencijalan
porast\parencite{oldfield2014}.  Da bi se razumela popularnost i sveža
perspektiva koju polje donosi neophodno je osvrnuti se na istoriju.

Fišerova\footnote{
} analogija o \keyword{bravi i ključu}\footnote{
} ponovo otkrivena nezavisnim
istraživanjima naučnika Hsien Wu,  Mirski i Paulinga (pedesetih godina prošlog
veka) postavila je temelje ''opšteprihvaćene'' \keyword{struktura-funkcija}
paradigme \parencite{dunker2001}.
% ''
% Specifične osobine nativnih\ref{} proteina mi pripisujemo njihovoj jedinstveno
% definisanoj konfiguraciji.  Denaturisane proteine mi smatramo okarakterisane
% izostankom jedinstveno definisane konfiguracije
% ''(Mirski i Pauling)
\en{“The characteristic specific properties of native\footnote{
    nativno stanje proteina je savijeno, operativno, funkcionalno stanje}
    proteins we attribute to their uniquely defined configurations. The
    denatured protein molecule we consider to be characterized by the absence
  of a uniquely defined configuration.”} (Mirski i Pauling)

Predloženi model prilagođen je ponašanju enzima, čija mogućnost da katalizuju
substrat\footnote{} zavisi od jasno definisanog oblika koji moraju da zauzmu
odnosno u koji moraju da se saviju\ref{}. Substrat (ključ ili funkcija) određuje
oblik enzima (brave ili strukture). Kontrapozicijom sledi da nedostatak
strukture vodi izostanku funkcije.

Prvi kontraprimer \en{serum albumin} (1950) ukazivao je da specifične
zahteve enzima ne treba generalizovati na sve proteine. Ipak brava-ključ model
i njena moderna varijanta \en{induced-fit theory} dominirale su prošlim vekom,
zanemarujuću konstantno rastući skup funkcionalnih ''ne-nativnih'' proteina
čije postojanje nisu mogle da objasne. Sa druge strane tehnološki napretci u
razlučivanju strukture proteina jasno su demonstrirali obimno postojanje
funkcionalnih proteina bez uređene 3D strukture od kojih su neki bili neuređeni
celom dužinom\parencite{dunker2001}.
Nova paradigma je bila neophodan.

Hipoteza proteinskog trojstva\parencite{dunker2001} (nastala tek početkom
stoleća) predlaže da funkcija proteina može zavisiti od bilo koje od tri
stanja ili tranzicije između tih stanja. Sva tri stanja predstavljaju
nativne oblike proteina i analogna su fazama materije na zemlji.
\begin{itemize}
  \item Uređen protein, čvrsto stanje
  \item \keyword{Topljiva globula} \en{molten globule}, tečno stanje
% Molten globule predstavlja parcijalno savijeno (tečno, liquid like) stanje proteina (Ptitsyn and Crane-Robinson)
  \item međustanje \keyword{Pre-topljiva globula} \en{pre-molten globule}. \\
        Usled nejasne tranzicije između stanja topljivog globula i nasumičnog klupka
        \parencite{2001} (suprotno analogiji tečnog i gasovitog stanja) model je
        dopunjen međustanjem.
  \item \keyword{Nasumično klupko} \en{random coil}, gasovito stanje
\end{itemize}

Povezanost sekvence sa strukturom sugeriše da je
neuređenost enkodirano inherentno svojstvo \parencite{dunker2001} stoga ove
proteine nazivamo \keyword{Inherentno Neuređeni Proteini} skraćeno
\keyword{IDP} a njiove neuređen ali funkcionalne regione
\keyword{IDPr}\parenctile{uversky2016}.


\subsection {Osobine i uticaj na funkciju}
Detaljno opisivanje osobina i posledica neuređenosti prevazilazi obim rada i
stručnost autora zalazeći u biohemiju. Iz tog razloga prezentujemo samo osobine
relevantne za ovaj rad kao i neka svojstva koja smatramo relevantnim za buduća
istraživanja.


% The word “intrinsically” indicates a sequencedependent characteristic 
% Mali molekuli, ligandi (6,7).
% makromolekulski binding partneri ili post translacione modifikacije (PTM)(7)
% mogu uticati na tranzicije između IDP i IDPr i struktuiranog proteina, domena. (8)
%
% Linkers, entropic springs ili elastomers, entropic bristles i nativni molten globule oslanjaju se na fleksibilnost zarad izvršavanja funkcije
% Order-Disorder mogu da budu uzročnici funkcije (underlie function???)
%   - savijanje pri spajanju
%   - odvijanje za chaperon aktivaciju
%
% Biološki procesi: alternative splicing,  prolazak kroz uske pore ili kanale,
%                   mngoe ali ne sve PTM, overprinting(dual coding), INDELS
%
%




\subsection{Eksperimentalno ispitivanje neuređenosti}

\item \keyword{Kristalografija X zracima} \en{X-ray crystallography}
\item Spektrosokopija Nuklearnom Magnetnom Rezonancom (NMR) \en{NMR spectroscopy}
\item \en{Circular dichroism (CD) spectroscopy}
\item \en{Protease digestion}
\item \en{Stoke’s radius determination}



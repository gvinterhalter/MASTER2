
\chapter{Uvod} % Main chapter title

\label{Uvod} % For referencing 


%------------------------------------------------------------------------------

\section{Osnovni biološki i hemiski koncepti}

Svaki živi organizam sastoji se od jedne ili više ćelija a svaka ćelija od
molekula. Veliki \footnote{ Obično se molekulska masa od $1000 Da$ (Daltona) uzima kao 
granica između malih molekula i makromolekula.}
molekuli (makromolekuli) biološkog porekla obično \footnote{
  Lipidi recimo nisu polimeri ali su principijalno slični
} su sačinjeni od
ponavljajućih strukturnih jedinica \keyword{monomera} \textit{(mono- = jedan,
mer- = deo)}, međusobno povezanih \keyword{kovalentnim} vezama.  Takav molekul
zovemo \keyword{polimer} \textit{(poli- =mnogo, -mer= deo)}. 
% Polimer može da bude \textit{homo-polimer}, sačinjen od jednog tipa monomera
% ili suprotno \textit{hetero-polimer}, sačinjen od nekoliko raznih tipova
% monomera.
Skup monomera možemo da smatramo azbukom koja gradi jezik polimera.  Mali broj
monomera je dovoljan za strukturnu kompleksnost bilo koje ćelije.  Tri 
najznačajnija tipa bioloških polimera i njihovi monomeri prikazani su tableom \ref{tab:polimeri}.


\begin{table}[htpb]
  \centering
  \caption{Najznačajniji biološki polimeri}
  \label{tab:polimeri}
  \begin{tabular}{ll}
    \keyword{Polimer}            & \keyword{Monomer} \\
    Ugljeni hidrati              & Monosaharid (šećeri) \\
    Nukleinska kiselina (DNK)    & Nukleotid \\
    Protein                      & Aminokiselina \\
    \hline



  
  \end{tabular}
\end{table}



\subsection{Proteini}

Proteini su najčešći biološki makromolekuli koji čine i do $80\%$ suve mase organizma.
Strukturno protein je linearan polimeri sačinjen od lanca \keyword{aminokiselina}. 




\section{Tehnike predvidjanja neuredjenosti}



\section{Funkcija proteina}

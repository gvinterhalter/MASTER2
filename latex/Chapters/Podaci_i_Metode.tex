
\chapter{Podaci i metode} % Main chapter title

\label{Podaci i metode} % For referencing 



\section {Podaci}

Za metode koje prezentujemo potrebne su tri vrste informacija:
\begin{enumerate}
  \item Što više različitih proteina
  \item Pouzdana anotacija funkcija
  \item Informacije o funkcijama, prvenstveno međurelacije\\
    (Međurelacije između funkcija su bitne  ako je potrebno grupisati ih)
\end{enumerate}


\subsection{Podaci iz originalnog rada}

U originalnom radu \parencite{Xie2007} korišćena je  baza 
\keyword{Svis-Prot} Poglavlje \ref{svis-prot}, verzija 48 iz 2005.
Verzija 48 ima 201 560 proteina od kojih 196 326 imaju dužinu preko 40
aminokiselina (što je potrebno zbog Definicije \ref{pdis_def} u nastavku). Funkcije
pridružene proteinima izražene su \keyword{kontrolisanim vokabularom}
\en{controlled vocabulary} koga čine takozvane UniProtKB \keyword{ključne reči}
\en{keywords}. U verziji 48, UniProtKB sadrži 874 ključnih reči.  Zbog
statističke značajnost posmatrane su one ključne reči kojima je bilo anotirano
barem 20 proteina, tj. 710 ključnih reči.

Kao što smo pomenulli u Poglavlju \ref{svis-prot} kanonske sekvence(proteini) u Svis-Prot
bazi ''nisu redundantne'' u smislu da jedan unos u bazi predstavlja produkt
jednog gena iz jedne vrste organizma. Međutim za analizu funkcija
Svis-Prot \keyword{jeste statistički redundantna} \parencite{proveriti} jer
sadrže veliku količinu \keyword{homologih} proteina (prvenstveno ortologa).
Autori rada \parencite{Xie2007} su izvršili klasterovanje Svis-Prot
proteina u \keyword{proteinske familije} dobivši 27 217 familija. Pri klasterovanju svaki protein
ima težinu kojom doprinosi daljoj analizi. Težina svakog proteina u preseku
klastera sa datom funkcijom je inverzno proporcionalna veličini preseka
tako da je zbir težina svih proteina jednaka veličini preseka.

% \textbf{komentar:} \\
% Ono što autori nisu elaborirali jeste da početni uslov od minimum 20 proteina
% po ključnoj reči možda nije dovoljan. Ako pretpostavimo zarad ilustracije
% normalnu raspodelu veličina klastera proteina, očekivali bi da klaster najčešće
% sadrži 7 proteina. Dakle iako je 50 proteina pridruženo nekoj funkciji ona
% verovatno ima pridruženih svega 7 familija proteina. Kako familija sadrži
% proteine pod pretpostavkom istog evolutivnog porekla njihova funkcija bi
% trebalo da je slična pa se onda postavlja pitanje da li je 7 familija dovoljno
% da bi se razmatrala data ključna reč. Ovo je primarno kritika za ključne reči
% jer one obično predstavljaju opšte pojmove.
%
% Sa druge strane za usko specijalizovane pojmove bila bi dovoljna jedna familija
% proteina jer bi ona predstavljala sve razne homologe (TODO Burkhard Rost,
% Termofili)

\subsection{Naši podaci}

U našem radu korišćen je skup proteina preuzet sa \keyword{CAFA3} takmičenja.
Ovaj skup je namenjen da bude trening skup za predikciju funkcija proteina \parencite{CAFA}. 

CAFA3 trening skup je pažljivo odabran podskup Svis-Prot baze (iz 2016.) koji
uključuje sve proteine iz model organizama: \en{Human, Mouse, Rat, S.
cerevisiae, S. pombe, E. coli, A. thaliana, Dictyostelium discoideum,
Zebrafish, \& Bacillus cereus. } sa izuzetokm sekvenci \en{Drosophila and
Candida} koje su preuzete iz svojih respektivnih genomskih baza (Lična
komunikacija sa Iddo Friedberg, PhD iz CAFA tima)

Iako ovaj pristup potencijalno proizvodi skup koji je \keyword{statisički
redundantan} u našoj analizi smo pretpostavili da to nije slučaj jer je čin
klasterovanja veoma računarski zahtevan, a nismo ubeđeni da je neophodan za
ovaj konkretan skup. Iz tog razloga u daljoj analizi predstavljamo uprošćenu
verziju formule koja ne uračunva težinu pojedinačnog proteina.

Svis-Prot proteini su kodirani jednim karakterom koristeći \keyword{IUPAC}
kodove.  U podacima postoje sekvence sa nestandardnim aminokiselinam 'U' i 'O'
ili višeznačnim oznakama 'B', 'J', 'X' i 'Z'.  Ovakve sekvence nisu podržane od
strane izabranog prediktora i za nas predstavljaju nevalidne proteinske sekvence. Pod
\keyword{validnom proteinskom sekvencom} smatraćemo sekvencu koja je validan
ulaz za prediktor, tj. sačinjena je od  zbuke od 20 standardnih aminokiselina i
ima najmanju dužinu 9\footnote{ Dužina 9 je minimum za VSL2b prediktor koji
koristimo}.

CAFA3 Podaci se sastoje od dve datoteke:
\begin{enumerate}
  \item \file{uniprot\_sprot\_exp.fasta}  sadrži 66 841 protein od kojih 66 599
    za našu analizu predstavljaju validnu proteinsku sekvencu. Od preostalih
    proteina 66.063 ima dužinu veću ili jednaku od 40 aminokiselina.
  \item \file{uniprot\_sprot\_exp.txt} pridružuje funkcije u obliku 
    \keyword{GO termina}. Zastupljeni su termini iz sva tri imenska prostora:
    16.117 ćelijskih komponenti, 5 966 molekulskih funkcija i 16 117 bioloških
    procesa. Jednom proteinu može biti pridruženo više GO termina i obrnuto.
\end{enumerate}

Naša analiza primarno je orijentisna na korišćenje GO termina za opis funkcije
i razlikuje se od originalnog pristupa.  Analiza sa GO terminima (grupisanje po
funkciji) zahteva prvenstveno poznavanje \textit{IS\_A} roditeljske veze između
termina. Takođe tokom istraživanja bile su nam potebne i ostale informacije o
terminima. Pomenute informacije dobili smo iz
\url{http://purl.obolibrary.org/obo/go.obo} dokumenta verzije 2017-12-01.

Radi poređenja dobijenih rezultata potrebno je poznavanje relacije između
ključnih reči i GO termina. Postoje dva dostupna mapiranja:
\begin{itemize}
  \item \url{www.uniprot.org/docs/keywlist.txt} verzija 20.12.2017 sadrži
    detaljan opis 1188 ključnih reči od kojih 195 pripada kategoriji
    \keyword{Molekulsih funkcija}.  Od 195 samo 145 ima mapiranje na jedan ili
    više GO termina.
  \item \url{ttp://geneontology.org/external2go/uniprotkb_kw2go} sadrži samo
    mapiranja i generiše ih \keyword{GOA projekat} \parencite{Barrell2009}.
    Ipak ova mapiranja nisu korišćena jer ... \keyword{TODO}
\end{itemize}

Pošto je originalni rad \parencite{Xie2007} iz 2007. godine postoji razlike u
vokabularu ključnih reči, razlike u samim sekvencama proteina, broj proteina i
razume se anotacije ključnih reči na proteine.  Iz tog razloga bilo je potrebno
prvo ponoviti analizu sa vokabularom ključnih reči da bi se procenilo koliko
ove razlike utiču na originalne rezultate ovog rada \parencite{Xie2007}.

Iz tog razloga CAFA3 podatke ne možemo da posmatramo kao crnu kutiju već je
bilo neophodno povezati ih sa Svis-Prot proteinima prvenstveno zbog
pridruživanja ključnih reči. Kako postoje razlike između najnovije verzije Svis-Prot
baze i CAFA3 podataka bilo je neophodno izvršiti ''korektno'' spajanje i
analizu razlika.  Informacija o pridruženim ključnim rečima takođe su nam bile
značajne za proveru validnosti mapiranja na GO termine i testiranje
potencijalnih drugih metoda mapiranja. Naša očekivanja su da iste funkcije
podrazumevaju anotaciju na iste proteine.  Ovi koraci detaljno su opisani u
Poglavnju \ref{Priprema_podataka}.



\section {Metod}

Cilj rada je ispitivanje veze između molekulske funkcije proteina i njegove
(ne)uređenosti tj. da li molekulska funkcija zavisi više od uređenosti ili
neuređenosti.

\textbf{Idealan slučaj.} 
Pretpostavimo da za proizvoljnu molekulsku funkciju znamo sve strukturno različite
proteine koji je obavljaju.  Da bi dali korektan odgovor  moramo da znamo kako
neuređenost pojedinačnog proteina utiče na ponašanje protein.  Zatim moramo da
znamo da li i kako to ponašanje (tip neurđenosti) utiče na datu funkciju.
% Ova vrsta znanja nije bila dostupna 2007. osim za jako mali skup proteina.

\textbf{Realnost.} 
\begin{itemize}
  \item Broj eksperimentalno određenih neuređenih regiona je veoma mali.
    \keyword{Disprot baza} eksperimentalno utvrđenih neuređenih regiona ima
    svega 803 proteina sa opisanih 2167 neuređenih regiona \parencite{disprot7}.
    Još gore pouzdanost ovih regiona je diskutabilna jer različite
    eksperimentalne tehnike koje su korišćene imaju različitu pouzdanost.
    Najveću pouzdanost nose regioni koji su eksperimentlano utvrđeni sa 
    većim brojem eksperimentalnih tehnika\footnote{Nisu ni sve eksperimentalne
    tehnike podjednako pouzdane} \parencite{disprot7}. 
  \item Prediktori su trenirani na malom podskupu proteina iz Disprot i PDB  baze.
    Čak i konsenzus nekoliko različitih prediktora ne daje dovoljno pouzdane
    rezultate o lokaciji neuređenog regiona \parencite{Mitic}.
  \item 
    Pozitivna strana je najnoviji napredak, razvoj prediktora koji direktno
    pokušavaju da predvide funkciju koju IDPr obavlja \parencite{meng_c20017}.
  % \item Koliko nam je poznato trenutno nema prediktora koji predviđaju tip
  %   neuređenosti. Takođe Opisivanje tipa neuređenosti predstavlja poduhvat.
  %   \begin{itemize}
  %     \item Prof. Vladimir Uverski predlaže nekoliko imena za opis različitih
  %       ponašanja neuređenih proteina \en{Folldon, Unfolldon, ...}
  %       \parencite{Uversky2017}.
  %     \item Prof. Peter Tompa je zaslužan za kreiranje 3 ontologije neuređenih
  %       regiona za bazu Disprot koji precizno modeluje tipove
  %       neuređenosti \parencite{disprot}.  .  Međutim koliko nam je poznato
  %       prediktori koji predviđaju termine ove ontologije još nisu napravljeni
  %   \end{itemize}

\end{itemize}

Jednostavna alternativa je da se pretpostavi da veći udeo neuređenih u odnosu na
uređene proteine podrazumeva da funkcija zavisi više od neuređenosti.  Dakle
izjednačavamo uzročnost \en{causation} i \keyword{korelaciju}. Međutim prvo je
potrebno definisati kada protein smatramo neuređenim.  Definicija mora da
ima biološkog smisla, da bude prilagođena analizi, ali pored takođe ograničena je
sposobnostima i preciznošću prediktora koji se korist.  Više o tome u nastavku.

\subsection{Predikcija neuređenosti proteina}

Autori \parencite{Xie2007} koristili su \keyword{PONDR VL3E} prediktor koji
postiže tačnost od $~87\%$ pri unakrsnoj validaciji nad uravnoteženim test
skupom.  Zbog ekonomičnosti i dostupnosti u našem radu korišćen je noviji
prediktor druge generacije \keyword{PONDR VSL2b}.
Relevantne karakteristike VSL2b detaljno su opisane u \ref{VSL2b}.
Za potrebe analize autori \parencite{Xie2007} uvode sledeću definiciju:

\newtheorem{mydef}{Definicija}
\begin{mydef}
\label{pdis_def}
Protein je \keyword{putativno neuređen}(najverovatno neuređen) \en {putatively disordered}
ako sadrži bar jedan region veći ili jednak od 40 uzastopnih aminokiselina
takvih da imaju \textit{predviđenu neuređenost} iznad 0.5. 
\end{mydef}

Onda definišemo operator $d$ takav da za svaku proteinsku sekvencu $s_i$ važi:

\[   
  d(s_i) = 
    \begin{cases}
      1 & \text{ako je} \quad s_i \quad \keyword{putativno neuređena}\\
      0 & \text{suprotno}
    \end{cases}
\]

Uslov ''$\ge40$'' u originalnom radu delom je posledica ograničenja VL3
prediktora koji je treniran na \keyword{dugim} sekvencama\footnote{L označava
duge regione, $\ge$ 30 AK}. Mi nismo u obavezi da sledimo ovo pravilo, ali ga sledimo
radi upoređivanja rezultata.

\subsection{Zavisnost dužine proteina i predikcije dugačkog neuređenog regiona}

Verovatnoća da po gornjoj definiciji protein bude klasifikovan kao verovatno
neuređen raste sa porastom njegove dužine. Ovo je ozbiljan problem koji utiče
na statističku značajnost rezultata. Autori \parencite{Xie2007} 
predlažu narednu formulu da se ta verovatnoća proceni:

Neka je $S_L$ skup proteina sa dužinama između $[L-l, L+l]$ gde je $l
= 0.1*L$. Dobijamo sledeće formule:

$$ S_L = \{s_i \mid \quad | L -  \Vert s_i \Vert | <= l \quad   \}$$
$$ P_L = \dfrac{ \sum_{s_i \in S_L} d(s_i)} {\Vert S_L \Vert}$$

Ponašanje $P_L$ predstavljeno je na Slici \ref{fig:PL1}. Glatkoća rezulatata
kontroliše se veličinom $l$ koja predstavlja prozor uprosečavanja. Kako prozor
uglačavanja raste sa porastom dužine proteina $(l = 0.1*L)$
tako da prozor uprosečavanja raste sa porastom dužine proteina te $P_L$
postaje glađe sa veličinom proteina. Konstantni prozor uprosečavanja bi bila
tehnika još poznata kao \en{rolling average} ili \en{boxcar filter} i
predstavljala bi prostu vrstu konvolucije. \keyword{Trenutno ne znamo zašto se
autor odlučio da veličina prozora raste sa dužinom proteina???}.


\begin{figure}[th]
\centering
\includegraphics[]{plots/PL_F}
\decoRule
\caption {
 Punom linijom predstavljena je $P_L$ sa prozorom uprosečavanja $l = 0.1L$,
 a krstići predstavljaju sirove vrednosti $l = 0$ 
}
\label{fig:PL1}
\end{figure}


Pored gore prikazanog 'originalnog' metoda predstavljamo još jedan pristup
\keyword{Slučajno generisani} \en{random generated} proteini za procenu $P_L$.
Razmotrićemo dva modela. Prvi je naivni model \keyword{uniformne verovatnoće}
koji podrazumeva da se svaka aminokiselina javlja sa istom verovatnoćom odnosno
$1/20$. U statistici ovo je još poznato kao \en{equiprobable model}.  Drugi
model koji ćemo zvati 'slučajni' ili 'random' model predstavlja slučajnu
promenljivu čija verovatnoća zavisi od učestalosti aminokiselina iz CAFA3 skupa
i prikazana je na Sliku \ref{fig:AK_ucestalost}.  Koristeći ova dva modela za
svaki protein generisan je slučajan protein iste dužine koji se koristi za
procenu $P_L$.


\begin{figure}[th]
\centering
\includegraphics[]{plots/AK_ucestalost}
\decoRule
\caption{Slučajni i uniformni modeli za procenu $P_L$}
\label{fig:AK_ucestalost}
\end{figure}



Poređenje ova dva pristupa sa originalnim $P_L$ prikazano je na Slici
\ref{fig:PL2}.  Originalni $P_L$ ostaje prikazan kao puna linija. Jasno se vidi
da slučajni model prikazan isprekidanom linijom predstavlja vizuelno dobru
aproksimaciju dok uniformni model verovatnoća prikazan tačkicama znatno odstupa
i dosta sporije raste (naizgled skoro linearno). Kako VSL2b predkitor
prepoznaje neuređene regione na osnovu učestalosti aminokiselina ovo ponašanje
nije čudno jer je manja verovatnoća pojave aminokiselina koje promovišu
neuređenost. Zbog suviše velikog odstupanja uniformni model nije korišćen u
daljoj analizi.


\begin{figure}[th]
\centering
% \includegraphics[scale=0.65]{Figures/PL2}
\includegraphics[]{plots/PL_F_cmp}
\decoRule
\caption{Različiti pristupi za procenu $P_L$}
\label{fig:PL2}
\end{figure}


Jedno od objašnjenja zašto je uniformni model naivan i toliko odstupa od
prvobitnog metoda proizilazi iz činjenice da aminokiseline imaju inherentno
različite verovatnoće. Naime, aminokiseline ne mogu  imati istu
verovatnoću jer se  broj njihovih kodona razlikuje. Neke aminokiseline
su kodirane sa samo jednim, a druge i sa 6 kodona. Očekivali bi da broj kodona
povećava učestalost aminokiseline i ta korelacija uz izuzetke arginina se vidi
na Slici \ref{fig:aminoacid} \parencite{AKfrekvencija}.

\begin{figure}[th]
\centering
\includegraphics[scale=0.7]{aminoacid}
\decoRule
\caption{Očekivana i realna učestalost  aminokiselina kod sisara\\ \footnotesize
(Preuzeto sa: \url{www.tiem.utk.edu/~gross/bioed/webmodules/aminoacid.htm})}
\label{fig:aminoacid}
\end{figure}



\subsection{Ocenjivanje zavisnosti funkcije od neuređenosti}

Neka je $S_j$ skup proteina koji imaj pridruženu funkciju $j$. Tada se procenat
putativno neuređenih proteina u oznaci $F_j$ može izračunati kao: $$F_j =
\dfrac{\sum_{s_i \in S_j} d(s_i)} {\Vert S_j \Vert} $$

Nultu hiptezu koja predviđa da je rezultat $F_j$ posledica samo slučajnosti, to
jest zavisi samo od $P_L$ opisana je preko slučajne veličine $Y_j$
gde je $X_L$ Bernulijeva slučajna veličina sa verovatnoćom $P(X_L = 0) = P_L$
odnosno $P(X_L = 1) = 1-P_L$

$$ Y_j = \dfrac {\sum_{s_i \in S_j} {X_{|s_j|}}}{\Vert S_j \Vert}$$

Ako $F_j$ izlazi iz intervala poverenja raspodele $Y_j$ onda funkcija $j$
sadrži značajno mnogo predviđenih neuređenih ili uređenih proteina. Preciznije
ako je \textit{p-value} $<0.05$ funkcija $j$ je povezana sa neuređenim
proteinima a ako je \textit{p-value} $>0.95$ funkcija $j$ je povezana sa
uređenim proteinima. Suprotno ne može ništa da se tvrdi za funkciju $j$.

$Y_j$ je teško izračunati analitički te mora da se pribegne empiriskom
računanju p-vrednosti. Empiriska p-vrednost određena je tako što je za 1000
realizacija $Y_j$ izračunato očekivanje da je realizacija $Y_j$ veća od $F_j$.
\begin{verbatim}
    p = np.array( [yj>Fj for yj in Yj_1000] ).mean()
\end{verbatim}
U radu \parencite{Xie2007} autori tvrde se da za veće skupove $S_j$ raspodela
$Y_j$ ponaša kao normalna pa se ocena Z-skor može dobiti kao
$Z_j=(F_j-\mu_j)/\delta_j$ gde je $\mu_j$ očekivanje a $\delta_j$ standardna
devijacija.  Dodatno p-vrednost može da se aproksimira kao $1/2(1-erf(Z_j/2))$
\footnote{$erf()$ je gausova funkcija greške,
$erf(x)=\dfrac{2}{\sqrt{\pi}} \int_{0}^{x}  e^{-t^2} dt$ }
ako raspodela liči na normalnu. Ovo je nekad korisno jer sa 1000 realizacija
$Y_j$ nema dovoljnu preciznost za p vrednost manju od $1/1000=0.001$. Međutim u
ovom radu to nije korišćeno jer su sva sortiranja (kao i u originalnom radu)
izvršena po Z-skor oceni.



\documentclass{beamer}

\usetheme{CambridgeUS}

\usepackage[utf8x,utf8]{inputenc} % make weird characters work
\usepackage[serbian]{babel}

\usepackage{graphicx}
\graphicspath{{../images/}}
\usepackage{color}

\usepackage{mathpazo} % Use the Palatino font by default
\usepackage{amsmath}

\usepackage{verbatim}

% -------------------------------------------------

% definišemo korisne komande
\newcommand{\en}[1]{(engl. \textit{#1})}
\newcommand{\lat}[1]{(latin. \textit{#1})}
\newcommand{\keyword}[1]{\textbf{#1}}
\newcommand{\ikeyword}[1]{\textit{\textbf{#1}}}
\newcommand{\tabhead}[1]{\textbf{#1}}
\newcommand{\code}[1]{\texttt{#1}}
\newcommand{\file}[1]{\texttt{\bfseries#1}}
\newcommand{\option}[1]{\texttt{\itshape#1}}


\newcommand{\uniprot}{\textit{UniProt} }
\newcommand{\uniprotkb}{\textit{UniProtKB} }
\newcommand{\swissprot}{\textit{Swiss-Prot} }
\newcommand{\trembl}{\textit{TrEMBL} }

\newcommand{\kw}[1]{\textbf{\textit{KW: #1}}}
\newcommand{\mf}[1]{\textbf{\textit{MF: #1}}}

\newtheorem{definicija}{Definicija}

% -------------------------------------------------

% \title{Bioinformatička analiza povezanosti funkcije i neuređenosti proteina}
\title{Bioinformatička analiza povezanosti funkcije i neuređenosti proteina}
% \subtitle{Master Rad}

\author{Goran Vinterhalter}
\institute{ Matematički fakultet, Beogradski univerzitet }
\date{\today}

% This is only inserted into the PDF information catalog. Can be left
% out. 

% If you have a file called "university-logo-filename.xxx", where xxx
% is a graphic format that can be processed by latex or pdflatex,
% resp., then you can add a logo as follows:

% \pgfdeclareimage[height=0.5cm]{university-logo}{university-logo-filename}
% \logo{\pgfuseimage{university-logo}}


% Delete this, if you do not want the table of contents to pop up at
% the beginning of each subsection:

% \AtBeginSubsection[]
% {
%   \begin{frame}<beamer>{Outline}
%     \tableofcontents[currentsection,currentsubsection]
%   \end{frame}
% }

% Let's get started
\begin{document}

\begin{frame}
  \titlepage
\end{frame}

% \begin{frame}{Outline}
%   \tableofcontents
%   % You might wish to add the option [pausesections]
% \end{frame}


% \section{Osnovni pojmovi}
% \subsection{Centralna dogma molekularne biologije}
% \subsection{Homologija}
% \section{Proteini}
% \subsection{Aminokiseline}
% \subsection{Struktura proteina}
% \subsection{Enzimi}
% \subsection{Funkcija proteina}
%
% \section{Inherentno neuređeni proteini}
% \subsection{Osobine i uticaj na funkciju}
% \subsection{Eksperimentalno ispitivanje neuređenosti}
% \subsection{Predikcija neuređenosti}
% \subsection{PONDR familija prediktora i VSL2b}
%
% \section{Baze podataka u bioinformatici} 
% \section{Ontologije gena}
% \subsection{GO-termini}
% \subsection{Relacije između GO-termina}
% \subsection{Molekulska funkcija}
% \section{UniProtKB/Swiss-Prot}

\section{Podaci}

\begin{frame}
  \frametitle{Podaci iz referentnog rada}
\end{frame}


\begin{frame}
  \frametitle{Podaci korišćeni u ovom istraživanju}
\end{frame}


% Section and subsections will appear in the presentation overview
% and table of contents.
\section{Jupiter Notes}

\subsection{Uvod}

\begin{frame}{Šta je to Jupiter Notes ???}

  \quad Jupiter Notes (engl. \textit{Jupyter Notebook}) je veb aplikacija za
  kreiranje i deljenje dokumenata koji pored uobičajenog tekstualnog sadržaja
  sadrže i programski kod. 
  \pause



  \begin{itemize}
  \item {
      Dokument se sastoji iz niza vertikalnih ćelija: \pause
      \begin{itemize}
        \item ćelije teksta (markdown + latex math)
        \item ćelije koda  (kod koji Jezgro izvršava)
        \item sirove ćelije
      \end{itemize}
      \pause
  }
  \item {
      Svaka ćelija se može izvršiti (osvežiti) zasebno.
      \pause
  }
  \item {
      Dokument se prikazuje (renderuje) kao html stranica. Zato izlaz izvršavanja može
      da bude predstavljen kao html kod. \pause
      \begin{itemize}
        \item Lepo formatiran izlaz
        \item slike, grafikoni (svg, png, jpg, ...)
        \item interaktivni sadržaj (java script)
      \end{itemize}

      \pause
  }
  \item{
      Ćelije podržavaju interakciju sa korisnikom kroz potražnju za ulazom.
    }
  \end{itemize}
\end{frame}

\subsection{Notes format}

% \begin{frame}{Format i konverzije}
%   \begin{itemize}
%   \item {
%       Interno dokument se čuva kao JSON dokument.
%       (.ipynb format) \pause 
%   }
%   \item{
%     \begin{figure}[h!]
%     \includegraphics[scale=1]{nbPrimer_croped.jpg}
%     \end{figure}
%     \pause 
%   }
%   \item {
%       Čuvaju se i izlazi ćelija \pause 
%   }
%   \item {   
%       Dokument se može eksportovati : 
%       {pdf (preko latexa)}, {html}, {html kao prezentacija},
%        {markdown}, {izvrni kod}
%       )
%   }
%   \end{itemize}
% \end{frame}
%
%
% \section {Arhitektura}
%
% \subsection {Uvod u Arhitekturu}
%
% \begin{frame}{Arhitektura}
% \begin{itemize}
%   \item{ arhitektura je klijent server. \pause 
%           \begin{figure}[h!]
%           \includegraphics[scale=0.4]{nbKomponente_2.png}
%           \end{figure}
%           \pause
%         }
% \item{ Notes server i Jezgro (engl. Kernel) čine serverski deo aplikacije \pause }
%   \item { Jezgro izvršava sadržaj ćelija koda. Jezgro obično podržava
%           samo jedan jezik (Pajton, Haskell, Julia, R, ...) \pause }
%   \item {Mi želimo Jezgro koje će izvršavati C++ kod. }
% \end{itemize}
%
% \end{frame}
%
% \begin{frame}[fragile]{Dužnost Jezgra}
%   \begin{itemize}
%     \item {
%         Jezgro komunicira sa Notes Server applikacijom i mora da ispoštuje
%         određen format komunikacije (propisane poruke, sokete) \pause
%       }
%     \item{ Struktura poruke:
%         \small
% \begin{verbatim}
% {
%   'header' :{
%     'msg' : uuid,
%     'username' : str,
%     'session' : uuid,
%     'msg_type' : str,
%     'version' : 5.0
%   }
%
%   'parent_header' : dict,
%   'metadata' : dict,
%   'content' : dict
% }
% \end{verbatim}
%       }
%   \end{itemize}
% \end{frame}
%
% \begin{frame}[fragile]{Bitne poruke}
%   \begin{itemize}
%     \item{ šel ROUTER/DEALER soket
%         \begin{itemize}
%           \item { execute\_request / reply }
%           \item { introspect\_request / reply }
%           \item { completion\_request / reply }
%           \item { History\_request / reply }
%           \item { KernelInfo\_request / reply }
%           \item {...}
%         \end{itemize}
%       }
%
%     \item{ PUB/SUB soket
%         \begin{itemize}
%           \item { stream (stdout, stderr, ...) }
%           \item { display (html, svg, png, ..)}
%           \item { clearOutput }
%           \item { ... }
%         \end{itemize}
%       }
%
%     \item{ stdin ROUTER/DEALER soket
%         \begin{itemize}
%           \item { input\_request }
%         \end{itemize}
%       }
%
%     \item{ Dozvoljeno je dodavanje drugih vrsta poruka... }
%
%   \end{itemize}
% \end{frame}
%
% \section {C++ Jezgro}
%
% \begin{frame}[fragile]{Ideja za C++}
%   \begin{itemize}
%
%     \item{ Šta želimo?
%         \begin{itemize}
%           \item{Korisnik ima osećaj kao da radi sa skript jezikom}
%           \item{Tačnije, promene u jednoj ćeliju utiču na ostale }
%         \end{itemize}
%         \pause
%       }
%
%     \item{ Koristimo dinamičko učitavanje. Ćeliju izvršavamo tako što: \pause
%         \begin{itemize}
%           \item{Kompjaliramo: \verb|g++ -shared -fpic| \pause}
%           \item{U pythonu učitavamo .so preko CDLL f-je.
%             (dlopen u pozadini) \\
%             flagovi: \verb|RTLD_GLOBAL  RTLD_DEEPBIND|
%
%           \pause}
%           \item {pozivamo funkciju void \_\_run\_\_() koju je Jezgro prethodno
%               dodalo na kraj koda \pause}
%         \end{itemize}
%       }
%
%     \item{  \verb|void __run__()|  se automatski generiše
%          i sadrži izraze koji nisu deklaracije/definicije.
%          Da bi ih razlikovali koristmo specijalnu sintaksu koja
%         je inače standardna za Jupiter notes jezgra  \pause
%         \begin{itemize}
%           \item{ \verb|%r| komanda; - liniska magija, komanda je trenutno
%             jednoliniski izraz. U duhu C++ jezika to bi trebalo prepraviti.}
%           \item{ \verb|%%r| kod; -  ćeliska magija, cela ćelija se
%             smešta u \verb|void __run__()|}
%         \end{itemize}
%       }
%
%
%   \end{itemize}
%
%
% \end{frame}
%
% \begin{frame}[fragile] {Primer 1}
%
%   \fontsize{8pt}{0.2pt}
%
%
%   \begin{columns}
%     \column[t]{0.2cm}
% \begin{verbatim}
% 1.
% \end{verbatim}
%     \column[t]{4.5cm}
% \begin{verbatim}
% int a = 10;
% %r cout << a;
% void f() {cout << a+1 << endl;}
% \end{verbatim}
%     \column[t]{4.5cm}
% \begin{verbatim}
% int a = 10;
% void f() {cout << a+1 << endl;}
% void__run__(){ cout << a; }
% \end{verbatim}
%     \column[t]{1.5cm}
% \begin{verbatim}
% result:
% 10
% \end{verbatim}
%   \end{columns}
%
%
%     \begin{columns}
%     \column[t]{0.1cm}
% \begin{verbatim}
% 2.
% \end{verbatim}
%       \column[t]{4.5cm}
% \begin{verbatim}
% %rr
% a = 25;
% cout << a << endl;
% f();
% \end{verbatim}
%       \column[t]{4.5cm}
% \begin{verbatim}
% extern int a;
% void f();
% void__run__(){ 
%   a = 25;
%   cout << a << endl; 
%   f();
% }
% \end{verbatim}
%     \column[t]{1.5cm}
% \begin{verbatim}
% result:
% 25
% 26
% \end{verbatim}
%     \end{columns}
%
%
%     \begin{columns}
%     \column[t]{0.1cm}
% \begin{verbatim}
% 3.
% \end{verbatim}
%       \column[t]{4.5cm}
% \begin{verbatim}
% float a = 3.14;
% %r cout << a << endl;
% %r f();
% \end{verbatim}
%       \column[t]{4.5cm}
% \begin{verbatim}
% void f();
% float a = 3.14;
% void__run__(){ 
%   cout << a << endl;
%   f(); 
% }
% \end{verbatim}
%     \pause
%     \column[t]{1.5cm}
% \begin{verbatim}
% result:
% 3.14
% 26
% \end{verbatim}
%     \end{columns}
%     \pause
%
%
%
%     \begin{columns}
%     \column[t]{0.1cm}
% \begin{verbatim}
% 4.
% \end{verbatim}
%       \column[t]{4.5cm}
% \begin{verbatim}
% %rr
% cout << a << endl;
% f();
% \end{verbatim}
%       \column[t]{4.5cm}
% \begin{verbatim}
% extern float a;
% void f();
% void__run__(){ 
%   cout << a << endl; 
%   f();
% }
% \end{verbatim}
%     \pause
%     \column[t]{1.5cm}
% \begin{verbatim}
% result:
% 3.50325e-44
% 26
% \end{verbatim}
%     \end{columns}
% % \end{block}
%
% \end{frame}
%
% \begin{frame}[fragile]{Rešenje}
%   \begin{itemize}
%
%     \item{ Funkcije su pokazivači na funkcije
%         \pause
%         \begin{itemize}
%           \item{funkciju \verb|void f() {...}| zamenjujemo pokazivačem
%               tj. \verb|void f_1() {...} | \\ \verb|void (*f)() = f_1;|
%             }
%         \pause
%       \item{Promena ponašanja funkcije (bez menjana potpisa) se svodi
%         na promenu vrednosti  pokazivača.}
%         \end{itemize}
%         \pause
%       }
%
%     \item{ Sakrivanje simbola 
%         \begin{itemize}
%           \item{ Pravo ime sakrivamo od korisnika (dodajemo mu sufiks tj. verziju). \pause
%             }
%           \item{ ako dođe do redefinicije koja menja tip simbola (tip promenljive, potpis funkcije)
%             inkrementiramo verziju sufiksa. \pause
%             }
%           \item { Imenu je promenjen sufkis samo u trenutnoj ćeliji.
%               U starijim ćelijma se idalje referiše staro ime jer nisu osvežene.
%             } \pause
%           \item {
%               Osvežavanje stare ćelije može da proizvede grešku jer se potpis više ne poklapa
%               ili tip više ne podržava traženo ponašanje.
%             }
%         \end{itemize}
%       }
%
%
%   \end{itemize}
%
%
% \end{frame}
%
% \section{Clang}
%
% \subsection{libclang}
%
% \begin{frame}{libclang}
%
% \begin{itemize}
%   \item { \texttt{libclang} je C interfejs za Clang biblioteke sa malim API-jem koji pruža mogućnosti  parsiranja C++ koda.
%   }
%   \pause
%   \item { Interfejs je osmišljen tako da korisnik definiše funkcije koje se pozivaju
%       tokom prolaska kroz \textbf{apstraktno stablo sintakse} koje paresr generiše.
%   }
%   \pause
% \item { Za svaki čvor stabla (CXCursor) možemo dobiti informacije:
%       \begin{itemize}
%       \item {vrsta čvora}
%       \item{ datoteka u kojoj se nalazi }
%       \item{ Lokacija u izvornom kodu }
%       \item{ tip podataka (potipis, povratna vrednost) } 
%       \item{ ime simbola }
%       \item{ kvalifikacioni atributi }
%       \item{ referenca na definiciju }
%       \item{...}
%       \end{itemize}
%
%   }
% \end{itemize}
%
% \end{frame}
% %
% %
% % \begin{frame}[fragile]{libclang}
% %
% %
% %   Auto ?
% % \begin{itemize}
% %     \item{\verb|auto| je predstavljen kursorom 'unresolved'}
% %     \item{Međutim već naredni kursor sadrži potreban tip}
% %     \item{U slučaju \verb|const auto &|, prvi kursor je \verb|&|}
% % \end{itemize}
% %
% %   \scriptsize
% % \begin{verbatim}
% % int * * a = nullptr;                                                            
% % auto b = 3.14;                                                                  
% % auto f(int  c) {return nullptr};
% %   ----------------------------------------
% % TranslationUnitDecl <<invalid sloc>> <invalid sloc>
% % |-VarDecl <t1.cpp:2:1, col:13> col:9 a 'int **' cinit
% % | `-ImplicitCastExpr <col:13> 'int **' <NullToPointer>
% % |   `-CXXNullPtrLiteralExpr <col:13> 'nullptr_t'
% % |-VarDecl <line:4:1, col:10> col:6 b 'double':'double' cinit
% % | `-FloatingLiteral 0x25b0c58 <col:10> 'double' 3.140000e+00
% % |-FunctionDecl <line:5:1, col:31> col:6 f 'void* (int)'
% % | |-ParmVarDecl <col:8, col:13> col:13 c 'int'
% % | `-CompoundStmt <col:16, col:31>
% % |   `-ReturnStmt <col:17, col:24>
% % |     `-CXXNullPtrLiteralExpr <col:24> 'nullptr_t'
% % `-EmptyDecl <col:32> col:32
% % \end{verbatim}
% %
% % \end{frame}
%
%
%
%
%
%
% \section{Kraj}
%
% \begin{frame}{Kraj}
%   \begin {itemize}
% \item{ Šta još nismo implementirali:
%     \begin{itemize}
%     \item {Trenutno ne podržavamo klase/strukture, unije i šablone.}
%     \item {Segmentation Fault ubija jezgro.}
%     \item {Interaktivni  unos korisnika fali.}
%     \item {Nisu sve poruke implementirane.}
%   \end{itemize}
%     }
%
%
%   \item{ Literatura:
%       \begin{itemize}
% \item {
% Ipython Developer documentation \\
%     \scriptsize
% \url{https://ipython.org/ipython-doc/3/development/}
% }
%
% \item {
% Ipython magic documentation \\
%     \scriptsize
% \url{ https://ipython.org/ipython-doc/3/interactive/magics.html }
% }
%
% \item {
% dlopen man page 
%     \scriptsize
% \url{ http://linux.die.net/man/3/dlopen}
% }
%
%
% \item {
% Jupyter project 
%     \scriptsize
% \url{http://jupyter.org/}
% }
%
% \item {
% Clang documentation 
%     \scriptsize
% \url{http://clang.llvm.org/}
% }
%
% \item {
% Linkers and Loaders,  John R. Levine \\
%     \scriptsize
% % \url{http://www.amazon.com/Linkers-Kaufmann-Software-Engineering-Programming/dp/1558604960}
% }
%       \end{itemize}
% }
%   \end {itemize}
%   
% \end{frame}
%
%
\end{document}
